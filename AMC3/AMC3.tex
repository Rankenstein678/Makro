\documentclass{article}
\usepackage{graphicx}
\usepackage{float}
\usepackage{epstopdf}
\usepackage{mathptmx}
\usepackage{siunitx}
\usepackage[version=4]{mhchem}
\usepackage{booktabs}
\usepackage{rotating}
%\usepackage{chemstyle}
\usepackage{hyperref}
\usepackage{lscape}
\usepackage[ngerman] {babel} 
\usepackage{footnotehyper}
\makesavenoteenv{table}
\usepackage{csquotes}
\makesavenoteenv{tabular}
\usepackage{biblatex}
\addbibresource{AMC3.bib}

\sisetup{round-mode=none}

\usepackage{acronym}
\newcommand{\subtxt}[1]{_{\text{#1}}}
\author{}
\title{

{\Huge \uppercase {AMC3 G15 V2}}
\vfill
	 	\rule{\linewidth}{2pt} \\[6pt] 
	 	\huge AMC3 - Kinetik der Emulsionspolymerisation \\ 
	 	\rule{\linewidth}{2pt} 
	 \vfill
	
	
	\large
	Autoren: \hfill Jess Singrin, Lennart Appel, Phillip Zazzetta\\
	Gruppennummer: \hfill 15\\
	Mails: \hfill jess.singrin@rwth-aachen.de\\\hfill lennart.appel@rwth-aachen.de\\\hfill phillip.zazzetta@rwth-aachen.de \\
	Tag der Durchführung: \hfill 13.11.2025 \\
	Abgabe 1: \hfill     \\ %zu ergänzen
    Abgabe 2  \hfill    %zu ergänzen
}

	
	
\date{Dezember 2025}

\begin{document}
\section{Abkürzungsverzeichnis}
\begin{acronym}
\acro{PMMA}{Polymethylmetacrylat}
\acro{SDS}{Natriumlaurylsulfat}
\acro{CMC}{Kritische Mizellenkonzentration}
\acro{MMA}{Methylmethacrylat}
\acro{KPS}{Kaliumperoxodisulfat}
\acro{VE-Wasser}{vollentsalztes Wasser}
\end{acronym}
\clearpage



\subsection{Berechnung des Umsatzes}
Um den Umsatz zu berechnen stehen mehrere Wege offen. Im folgenden wird der Umsatz sowohl mithilfe gravimetrische Mittel, als auch über den 
Reaktionswärmestrom berechnet.
\subsubsection{Berechnung auf gravimetrischen Weg}
Zur Bestimmung des Umsatzes wurden während Reaktion 11 Proben entnohmen und die Masse an enthaltenem Polymer bestimmt (Tabelle \ref{tab:gravimetrie}). 
Zunächst wird die Masse des Polymers in Lösung bestimmt. Dazu wird die Probenmasse mit dem Verhältnis von Volumen von Lösung zu Probe multipliziert.

\begin{equation}
    m\subtxt{Polymer} = \frac{V\subtxt{Lösung}\cdot m\subtxt{Probe}}{V\subtxt{Probe}}
\end{equation}
Mit Probe 1 ergibt sich:
\begin{align*}
     m\subtxt{Polymer} &= \frac{(V\subtxt{Wasser}+ V\subtxt{Monomer}+V\subtxt{Initiator}+V\subtxt{Tensid}-V\subtxt{entnommenes Volumen})\cdot m\subtxt{Probe}}{V\subtxt{Probe}}\\
      &=\frac{(\SI{180}{\milli\liter}+ \SI{27}{\milli\liter}+\SI{5}{\milli\liter} +\SI{27}{\milli\liter}\footnote{Wie zuvor angegeben, wurde das genaue Volumen nicht bestimmt.} - \SI{0}{\milli\liter}
)\cdot \SI{0.041}{\gram}}{\SI{2}{\milli\liter}}\\
&= \SI{4.8995}{\gram}
\end{align*}
Zur Berechnung des Umsatzes wird weiterhin die Masse des eingesetztes Monomers benötigt.
\begin{align}
    m\subtxt{0, MMA} &= V\subtxt{MMA} \cdot \rho\subtxt{MMA}\\
    &= \SI{27}{\milli\liter}\cdot \SI{0.94}{\gram\per\milli\liter}\nonumber\\
    &= \SI{25.38}{\gram}\nonumber
\end{align}
Der Umsatz lässt sich über das Verhältnis von polymerisierter Masse zur ursprünglichen Monomermasse berechnen.
\begin{equation}
    X\subtxt{grav} = \frac{m\subtxt{Polymer}}{m\subtxt{0, MMA}}
\end{equation}
Für Probe 1 ergibt dies
\begin{align*}
    X\subtxt{grav} &= \frac{\SI{4.8995}{\gram}}{\SI{25.38}{\gram}}\\
    &= 0.193
\end{align*}
Analog wird für alle Proben verfahren. Eine Auftragung gegen die Reaktionszeit ist in Abbildung \ref{fig:umsatz_zeit} zu sehen.

\begin{figure}
    \centering
   % \includegraphics[width=.9\linewidth]{umsatz_zeit.pdf}
   \caption{Gravimetrisch bestimmter Umsatz im Reaktionsverlauf.}
   \label{fig:umsatz_zeit}
\end{figure}
\begin{sidewaystable}
    \centering
    \caption{Massen der Proben der gravimetrischen Umsatzbestimmung.}
    \sisetup{round-mode=figures}
\begin{tabular}{rccccc}
\toprule
 Messung & Reaktionszeit [\unit{\min}] &   Volumen der Lösung [mL] & Masse der Probe [\unit{\gram}]&Masse des Polymers [\unit{\gram}]               & Umsatz                                       \\
\midrule
  1 &  2 & 239 & \num[round-precision=2]{0.04100000000000037} & \num[round-precision=2]{4.899500000000044}  & \num[round-precision=2]{0.19304570527974957} \\
  2 &  4 & 237 & \num[round-precision=2]{0.04640000000000022} & \num[round-precision=2]{5.498400000000026}  & \num[round-precision=2]{0.21664302600472915} \\
  3 &  6 & 235 & \num[round-precision=2]{0.07699999999999996} & \num[round-precision=2]{9.047499999999996}  & \num[round-precision=2]{0.35648148148148134} \\
  4 &  8 & 233 & \num[round-precision=2]{0.10740000000000016} & \num[round-precision=2]{12.512100000000018} & \num[round-precision=2]{0.49299054373522533} \\
  5 & 10 & 231 & \num[round-precision=2]{0.1396999999999995}  & \num[round-precision=2]{16.135349999999942} & \num[round-precision=2]{0.6357505910165462}  \\
  6 & 12 & 229 & \num[round-precision=2]{0.1515999999999984}  & \num[round-precision=2]{17.35819999999982}  & \num[round-precision=2]{0.6839322301024358}  \\
  7 & 14 & 227 & \num[round-precision=2]{0.1700999999999997}  & \num[round-precision=2]{19.306349999999966} & \num[round-precision=2]{0.7606914893617008}  \\
  8 & 16 & 225 & \num[round-precision=2]{0.17249999999999943} & \num[round-precision=2]{19.406249999999936} & \num[round-precision=2]{0.7646276595744655}  \\
  9 & 18 & 223 & \num[round-precision=2]{0.18230000000000146} & \num[round-precision=2]{20.326450000000165} & \num[round-precision=2]{0.8008845547675401}  \\
 10 & 20 & 221 & \num[round-precision=2]{0.18100000000000094} & \num[round-precision=2]{20.000500000000102} & \num[round-precision=2]{0.7880417651694288}  \\
 11 & 25 & 219 & \num[round-precision=2]{0.1772000000000009}  & \num[round-precision=2]{19.4034000000001}   & \num[round-precision=2]{0.7645153664302641}  \\
\bottomrule
\end{tabular}
\label{tab:gravimetrie}
\end{sidewaystable}
 


\end{document}